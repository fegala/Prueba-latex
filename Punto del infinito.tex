\documentclass[12p]{article}
\usepackage[utf8]{inputenc}
\usepackage[spanish]{babel}
\usepackage{graphicx}

\begin{document}
\begin{titlepage}
\title{Límites y el punto del infinito}
\author{Fernando Garza Landa}
\maketitle
\vfill
\end{titlepage}

\begin{abstract}
Nota: El punto del infinito puede estudiarse dentro del plano real o dentro del plano complejo, en este trabajo, nos enfocaremos al concepto relacionado con el plano complejo tocando solo ligeramente su uso en el plano real.

Dentro de los límites, podemos encontrar el concepto de "punto del infinito" denotado por $\infty$, el cual vuelve al plano complejo un \textit{plano complejo extendido}. Este trabajo nos ayudará a entender mejor de qué trata este tema, extraído del capitulo 13 del libro \textit{Variable compleja y aplicaciones} de \textsc{Ruel V. Churchill}
\end{abstract}
\vspace{1cm}
\tableofcontents
\vspace{.5cm}
\listoffigures

\newpage

\section{Representación}
La manera más común de visualizar el punto del infinito es considerando una esfera unidad llamada \textit{Esfera de Riemann} cuyo centro z=0 y ecuador es atravesado por el plano complejo. 

\begin{figure}[h]
\begin{center}
\includegraphics[width=250pt]{esfera.png}\\
\caption{Esfera de Riemann}
\label{fig:figura1}
\end{center}
\end{figure}

A cada punto \textbf{P} de la superficie de esta esfera le corresponderá (conocida como \textit{proyección estereográfica}) un punto \textbf{z} del plano, donde el punto \textbf{P} se determina por la intersección de la recta que pasa por el polo norte \textbf{N} y el punto \textbf{z}. "De forma análoga, cada P sobre la esfera, excepto el polo norte N, corresponde exactamente a un punto \textbf{z} del plano. Haciendo corresponder el polo norte N al punto del infinito" \cite{1}, dando una correspondencia uno a uno entre los puntos \textbf{P} y \textbf{z} del plano complejo extendido.

\subsection{Esfera de Riemann}
Profundizando un poco en el concepto de la $"$Esfera de Riemann$"$ encontramos que la esfera es la representación geométrica de los \textbf{números complejos extendidos}, la cual consiste en los números complejos ordinarios en conjunción con el símbolo $\infty$ para representar el infinito.

Una ventaja de usar estos numeros complejos extendidos es la de permitir la división por cero en algunas circunstancias, obteniendo algo como $1/0=\infty$

\subsection{En el plano real}
Dentro del conjunto de los números reales el punto del infinito es una entidad topológica y geométrica que se introduce a modo de cierra o \textit{frontera infinita}.

Agregando el punto del infinito a la recta real, genera una curva cerrada conocida como \textbf{Recta proyectiva real}.

\begin{figure}[h]
\begin{center}
\includegraphics[width=70pt]{real.png}\\
\caption{Recta proyectiva real}
\label{fig:figura2}
\end{center}
\end{figure}

\section{Desarrollo}
Para cada $\varepsilon$ positivo pequeño, aquellos puntos del plano complejo exteriores al círculo $|$z$|$=1/$\varepsilon$ corresponden a puntos de la esfera próximos al N. Llamamos al conjunto $|$z$|$ $>$ 1/$\varepsilon$ un entorno de $\infty$.

Tenemos la afirmación:
\[
\lim_{z\rightarrow z_0}f(z)=w_0
\]

Si $z_0$ y/o $w_0$ son sustituidos por el punto del infinito, la afirmación queda como:

\[
\lim_{z\rightarrow z_0}f(z)=\infty
\]



Finalmente tenemos que

\[\lim_{z\rightarrow z_0}f(z)=\infty \quad \textbf{si y sólo sí} \quad \lim_{z\rightarrow z_0}\frac{1}{f(z)}=0
\]

\vspace{1cm}
\section{Ejemplos}
\subsection{Ejemplo 1}
\cite{2} Nótese que

\[\lim_{z\rightarrow -1} \frac{iz + 3}{z + 1}= \infty \quad \textbf{ya que} \quad \lim_{z\rightarrow -1}\frac{z + 1}{iz + 3}=0
\]

Si

\[
\lim_{z\rightarrow \infty}f(z)=w_0
\]

entonces, $\forall \enskip \varepsilon$ positivo $\exists$ un $\delta$ positivo tal que

\begin{equation}
|f(z) - w_0| < \varepsilon \quad \textbf{siempre que} \quad |z|>\frac{1}{\delta}
\end{equation}

Luego

\begin{equation}
\lim_{z\rightarrow \infty}f(z)=w_0 \quad \textbf{si y solo si} \quad \lim_{z\rightarrow 0}f(\frac{1}{z})=w_0
\end{equation}
\vspace{.5cm}
\subsection{Ejemplo 2}
Por (2)
 
 \[\lim_{z\rightarrow \infty} \frac{2z + i}{z + 1}= 2 \quad \textbf{ya que} \quad \lim_{z\rightarrow 0}\frac{(2/z) + i}{(1/z) + 1}= \lim_{z\rightarrow 0}\frac{2 + iz}{1 + z} = 2
\]


\vspace{8 cm}



\begin{thebibliography}{x}
\bibitem{1} \textsc{Ruel V. Churchill}, \textsc{James Ward Brown}, \textit{Variable compleja y aplicaciones}, \textit{5ta ed.} \textbf{Pag.44}
\bibitem{2} \textsc{Ruel V. Churchill}, \textsc{James Ward Brown}, \textit{Variable compleja y aplicaciones}, \textit{5ta ed.} \textbf{Pag.45,46}
\end{thebibliography}
\end{document}					